%%%%%%%%%%%%%%%%%%%%%%%%%%%%%%%%%%%%%%%%%
% Compact Academic CV
% LaTeX Template
% Version 1.0 (10/6/2012)
%
% This template has been downloaded from:
% http://www.LaTeXTemplates.com
%
% Original author:
% Dario Taraborelli (http://nitens.org/taraborelli/home)
%
% License:
% CC BY-NC-SA 3.0 (http://creativecommons.org/licenses/by-nc-sa/3.0/)
%
% Important:
% This template needs to be compiled using XeLaTeX
%
% Note: this template has the option to use the Hoefler Text font, see the
% font configurations section below for instructions on using this font
%
%%%%%%%%%%%%%%%%%%%%%%%%%%%%%%%%%%%%%%%%%

% TODO
	% dizer quantas publicações tenho em cada categoria
	% colocar a bold o meu nome


%----------------------------------------------------------------------------------------
%	PACKAGES AND OTHER DOCUMENT CONFIGURATIONS
%----------------------------------------------------------------------------------------

\documentclass[11pt, a4paper]{article} % Document font size and paper size

\usepackage{fontspec} % Allows the use of OpenType fonts

\usepackage{geometry} % Allows the configuration of document margins
\geometry{a4paper, textwidth=5.5in, textheight=9.5in, marginparsep=7pt, marginparwidth=.6in} % Document margin settings
\setlength\parindent{0in} % Remove paragraph indentation

\usepackage[usenames,dvipsnames]{xcolor} % Custom colors

\usepackage{sectsty} % Allows changing the font options for sections in a document
\usepackage[normalem]{ulem} % Custom underlining
\usepackage{xunicode} % Allows generation of unicode characters from accented glyphs
\defaultfontfeatures{Mapping=tex-text} % Converts LaTeX specials (``quotes'' --- dashes etc.) to unicode

\usepackage{marginnote} % For margin years
\newcommand{\years}[1]{\marginnote{\scriptsize #1}} % New command for including margin years
\renewcommand*{\raggedleftmarginnote}{}
\setlength{\marginparsep}{7pt} % Slightly increase the distance of the margin years from the contant
\reversemarginpar

\usepackage[xetex, bookmarks, colorlinks, breaklinks, pdftitle={Francisco Nunes - CV},pdfauthor={Francisco Nunes}]{hyperref} % PDF setup - set your name and the title of the document to be incorporated into the final PDF file meta-information
\hypersetup{linkcolor=blue,citecolor=blue,filecolor=black,urlcolor=MidnightBlue} % Link colors

\usepackage{eurosym}


%----------------------------------------------------------------------------------------
%	FONT CONFIGURATIONS
%----------------------------------------------------------------------------------------

% Two font choices are available in this template, the default is Linux Libertine, available for free at: http://www.linuxlibertine.org while the secondary choice is Hoefler Text which comes bundled with Mac OSX.
% To use Hoefler Text, comment out the Linux Libertine block below and uncomment the Hoefler Text block. You will also need to replace the "\&" characters with "\amper{}" in section titles.

% Linux Libertine Font (default)
\setromanfont [Ligatures={Common}, Numbers={OldStyle}, Variant=01]{Linux Libertine O} % Main text font
\setmonofont[Scale=0.8]{Monaco} % Set mono font (e.g. phone numbers)
\sectionfont{\mdseries\upshape\Large} % Set font options for sections
\subsectionfont{\mdseries\scshape\normalsize} % Set font options for subsections
\subsubsectionfont{\mdseries\upshape\large} % Set font options for subsubsections
\chardef\&="E050 % Custom ampersand character

% Hoefler Text Font (bundled with Mac OSX)
%\setromanfont [Ligatures={Common}, Numbers={OldStyle}]{Hoefler Text} % Main text font
%\setmonofont[Scale=0.8]{Monaco} % Set mono font (e.g. phone numbers)
%\setsansfont[Scale=0.9]{Optima Regular} % Set sans font, used in the main name and titles in the document
%\newcommand{\amper}{{\fontspec[Scale=.95]{Hoefler Text}\selectfont\itshape\&}} % Custom ampersand character
%\sectionfont{\sffamily\mdseries\large\underline} % Set font options for sections
%\subsectionfont{\rmfamily\mdseries\scshape\normalsize} % Set font options for subsections
%\subsubsectionfont{\rmfamily\bfseries\upshape\normalsize} % Set font options for subsubsections

%----------------------------------------------------------------------------------------

\begin{document}

%----------------------------------------------------------------------------------------
%	CONTACT AND GENERAL INFORMATION SECTION
%----------------------------------------------------------------------------------------

{\LARGE Francisco Nunes}\\[1cm] % Your name
% Rua Dr. Lopes Cardoso 505\\ % Your address
% 4420 Gondomar, Portugal\\[.2cm]
Born: 1987---Porto, Portugal\\ % Your date of birth
Nationality: Portuguese \& Brazilian \\% Your nationality

Email: \href{mailto:francisco.nunes@fraunhofer.pt}{francisco.nunes@fraunhofer.pt}\\ % Your email address
Personal page: \href{http://francisconunes.me}{http://francisconunes.me}\\
Scholar: \href{https://scholar.google.com/citations?user=D2rDc98AAAAJ}{https://scholar.google.com/citations?user=D2rDc98AAAAJ}\\
Mastodon: \href{https://hci.social/@fnunes}{fnunes@hci.social}
% Twitter: \href{https://twitter.com/xico_nunes}{@xico\_nunes}, \href{https://twitter.com/fundedphdinhci}{@FundedPhDinHCI}

% \vfill % Whitespace between contact information and specific CV information

%------------------------------------------------



%------------------------------------------------

%\section*{Current position}

% \emph{PhD Student}, HCI Group, Vienna University of Technology % Your current or previous employment position

%------------------------------------------------

\section*{Research interests}
I am interested in learning about people and how they use technology. My projects are usually at the intersection of computer science, health sciences, design, and social sciences. Broad keywords for my work would be: Human-Computer Interaction, Computer-Supported Cooperative Work, Health Informatics, and Science and Technology Studies.


%----------------------------------------------------------------------------------------
%	EDUCATION SECTION
%----------------------------------------------------------------------------------------

\section*{Education}

\years{2020-2022}Master course in Academic and Clinical Education, University of Porto\\
\years{2012-2017}PhD in Informatics (Human-Computer Interaction), TU Wien\\
\textit{Thesis: ``The Everyday Life with Parkinson's and Self-care Technologies"}\\
% Supervised by \\ Geraldine Fitzpatrick.}\\
\years{2005-2010}Master in Informatics and Computing Engineering, University of Porto\\
\textit{Thesis: ``Healthcare TV Based User Interfaces for Older Adults"}
 %Supervised by Teresa Galvão and Paula Alexandra Silva.}


%----------------------------------------------------------------------------------------
%	WORK EXPERIENCE SECTION
%----------------------------------------------------------------------------------------

\section*{Experience}

\years{2019-now}Invited Assistant Professor, University of Porto, Porto\\
\years{2017-now}Senior researcher, Fraunhofer Portugal AICOS, Porto\\
\years{2016-2017}Researcher, NOVA-LINCS, Universidade Nova de Lisboa, Lisboa\\ % 5/2016
\years{2012-2017}PhD Student, Institute for Design and Assessment of Technology, TU Wien, Vienna\\ % 9/2012
\years{2008-2012}Junior researcher, Fraunhofer Portugal AICOS, Porto % 9/2010--7/2012
%\years{2008-2010}Research Assistant, Fraunhofer Portugal AICOS, Porto % 7/2008--8/2010



%----------------------------------------------------------------------------------------
%	GRANTS, HONORS AND AWARDS SECTION
%----------------------------------------------------------------------------------------

\section*{Grants, honours \& awards}

\subsection*{Acquired project grants (\euro{} values for my institution)}
\years{2022}\euro174K - ``Design and validation of ultra-personalised nutritional system to support the self-care of women with gestational diabetes'' funded by FCT (PT) -- \textit{PI.}\\
\years{2021}\euro106K - ``Adaptation, development, and deployment of a chatbot to improve health literacy and care of first-time parents'' funded by Aga Khan Dev. Network  \& FCT (PT) -- \textit{PI.}\\
\years{2020}\euro190K - ``Mobile Patient-centred System to Improve Drug Trials and Care of Older-adults with Rheumatic Diseases'' funded by H2020 AAL \& FCT (PT) -- \textit{PI.}\\
\years{2019}\euro117K - ``Demonstration of EyeFundusScope with Non-Expert Ophthalmology users'' funded by ANI (PT) -- \textit{Lead proposal writer and Co-PI.}\\
% \years{2011}\euro137K - ``Smartphones for Smart Seniors (S4S)'' funded by ANI (PT) -- \textit{Lead proposal writer.}\\
\years{*}\euro96K - Industry contracted research e.g., Alcatel-Lucent, ALMASCIENCE, Doist.
% Alcatel 49
% ALMASCIENCE 25
% Doist 2,4
% APPC Campainhas 20

\subsection*{Honours and awards}
\years{2022}Hounorable mention for Campainhas, Tech. Innovation prize Eng. Jaime Filipe\\
\years{2016}Two special recognitions for an `Excellent Review' at CHI 2016\\
%\years{2016}Grant for attending conference from TU Wien (\euro1390)\\
%\years{2015}Grant for attending CHI 2015 doctoral consortium (\$1300)\\
% \years{2012}Vienna PhD School of Informatics scholarship (3 years of funding)\\
\years{2012}Dance! Don't Fall was on the Top-25 Apps, CES Mobile Apps Showdown\\
\years{2011}Smart Companion was 2nd most creative app, ZON Creativity Award (prize: \euro6250)\\
\years{2010}Mover was 5th best Lifestyle app, Android Developer Challenge II (prize: equipment)

\section*{Teaching \& supervision experience}

\subsection*{Courses}

\years{2023-now}Human-Computer Interaction Laboratory (PhD @U. Porto)\\
\years{2023-now}Digital Health and data (Professor continuous education @Copenhagen U.) -- \textit{Invited lecture on self-care technologies}\\
\years{2021-2022}Media Laboratory (PhD @U. Porto)\\
\years{2019-2023}Learning from Users: User Research Methods for Technology Design (MSc/PhD @AICOS)\\
\years{2019-2022}Design Project Management and Methodology (Post-graduation @U. Porto)\\
\years{2016-2018}Qualitative methods and co-design of self-care technologies (MSc @Copenhagen U.) -- \textit{Invited lecture on self-care and self-care technologies}\\
\years{2015}HCI in Healthcare (MSc @TU Wien) -- \textit{Co-teaching and project supervision}\\
\years{2014}User Research Methods (MSc @TU Wien)-- \textit{Co-teaching and project supervision}


\subsection*{Supervised students}


\years{2023-now}Leonor Fonseca, PhD in Computational Media Design @U. Coimbra\\
\years{2023-now}Diana Liebetrau, MSc in IT @U. of Cape Town\\
\years{2023-now}Juliana Assis, MSc Multimedia Design @U. Coimbra\\
\years{2023-now}Pedro Garruço, MSc Multimedia Design @U. Coimbra\\
\years{2022}Petra Grego, MSc Multimedia Design @U. Coimbra\\
\years{2022}Rui Fernandes, MSc Informatics Engineering @U. Nova de Lisboa\\
\years{2020}Niloufar Chakhmaghi, MSc Media Informatics @TU Wien\\
\years{2019}Jo\~{a}o Sim\~{o}es, MSc Biomedical Engineering @U. Nova de Lisboa\\
\years{2018}Cristiana Braga, MSc Service Design @U. Porto\\
\years{2018}Jo\~{a}o Almeida, MSc Informatics Engineering @U. Porto\\
\years{2015}Martin Herndl, BSc Medical Informatics @TU Wien\\
\years{2012}João Cevada, MSc Informatics Engineering @U. Porto\\
\years{*}Other 8 students as unofficial MSc co-supervisor

% \years{ongoing}Niloufar Chakhmaghi, Co-design of web platform for Parkinson's consultations\\
% \years{2018}Cristiana Braga, Study of a self-management system for healthcare shiftworkers\\
% \years{2018}Jo\~{a}o Almeida, Restaurant recommendation system for people with food allergies\\
% \years{2015}Martin Herndl, Developing an Android Self-Management Application for People with Parkinson's Disease\\
% \years{2012}João Cevada, User Interface Design Recommendations for Fitting Smartphone Applications to Persons With Parkinson's\\

\subsection*{Supervised PhD internships}

\years{2019-2021}Cristina Mendes Santos, PhD Internet Interventions @Link\"{o}ping University\\
\years{2018}Martina {\v C}ai\'c, PhD Service Design @Maastricht University


\section*{Projects}
% MASPARK
% DEM

\subsection*{Selection of ongoing projects}

\years{2023-2025}\emph{NUTRIA} (FCT) -- is conducting user research and co-designing a meal recommendation system for women with gestational diabetes. I am the project PI.\\
\years{2022-2024}\emph{ParentCoach} (FCT) -- is conducting co-design workshops and adapting a chatbot for first-time parents in Portugal and South Africa. I am the project PI.\\
\years{2021-2024}\emph{COTIDIANA} (AAL) -- is creating a mobile solution to enable holistic and efficient patient monitoring for clinical care, clinical research, and drug trials, drawing on PROMS and passive smartphone sensing. I am the project PI. \href{https://cotidiana.eu/}{https://cotidiana.eu/}\\
% \years{2021-2023}\emph{ConnectedHealth} (ANI) -- is creating advanced information systems that support chronic patients, pharmacists, and hospital clinicians, in dealing with COPD, Diabetes, and Heart failure. I am leading the user research team at AICOS.\\
\years{2021-2023}\emph{HomeSenseALS} (FCT) -- is creating a mobile app to track Amyotrophic Lateral Sclerosis in clinical research field trials. I am leading the user research activities.
% \years{2019-2021}\emph{EyeFundusScopeNEO} (PT2020) -- This project is piloting EyeFundusScope, a solution to screen diabetic retinopathy, in CUF Infante Santo. I am currently the PI at AICOS.

% \subsection*{Selection of previous projects}

% \years{2014-2019}\emph{Clockwork} (EU AAL-JP) -- developed technologies to support the self-care of shift workers. I was in charge of supervising the user research, design, and evaluation of the smartphone app and the smart badge, the tasks of AICOS. I also coordinated the project consortium for 24 months. \href{http://clockworkproject.eu/}{http://clockworkproject.eu/}\\
%\years{2016-2019}\emph{Physio@Home} (PT2020) -- developed a solution to support physical rehabilitation for the home and clinic, using wearables and serious games. I was involved in the design and evaluation of the system; I also replaced the PI of the project for 14 months.
% \years{2015-2018}\emph{MASPARK} (La Marat\'{o} de TV3) -- This project developed a system to detect and address freezing of gait in patients with Parkinson's. I was PI for 15 months, in charge of supervising the design and evaluation of the system.\\
% \years{2011-2015}\emph{REMPARK} (EU ICT-FP7) -- This project developed a wearable system to monitor symptoms of people with Parkinson's, intervene in the gait, and collect data to improve the therapy. I participated in the first year of the project and was involved in user research and early design phases of the smartphone apps for patients. \href{http://www.rempark.eu/}{http://www.rempark.eu/}\\
% 11/2011-07/2012
% \years{2011-2014}\emph{S4S} (ANI) -- This project adapted the Windows Phone for older adults. I participated in the first year of the project and was involved in the user research, design, and evaluation of some of the smartphone applications.\\
% 10/2011-07/2012
%\years{2011-2012}\emph{Dance! Don't Fall} (Internal) -- This project created an Android application that monitors the risk of falling while actively reducing it through dancing. I participated in the user research, design, and evaluation of the application. \href{http://dancedontfall.projects.fraunhofer.pt}{http://dancedontfall.projects.fraunhofer.pt}\\
% \years{2010-2012}\emph{Smart Companion} (Internal) -- This project created an Android skin specifically designed for older people. I participated in the user research, design, and evaluation of the smartphone applications. Smart Companion is today being sold by GOCIETY under the name GOLIVEPHONE. \href{http://smartcompanion.projects.fraunhofer.pt}{http://smartcompanion.projects.fraunhofer.pt}\\
% \years{2009-2012}\emph{eCAALYX} (EU AAL-JP) -- This project created a telehealth system for older adults with chronic conditions. I participated in the the user research, design, and evaluation of the TV interfaces for older adults. \href{http://ecaalyx.org}{http://ecaalyx.org}\\
% 06/2009-07/2012
% \years{2009-2012}\emph{Mover} (Internal) -- This project created an Android app that explored the accelerometer to capture activity and thus promote movement. I participated in the design and evaluation of the app. Mover is today being sold as part of the GOLIVEPHONE mentioned above. \href{http://mover.projects.fraunhofer.pt/}{http://mover.projects.fraunhofer.pt/}


\section*{Service}

\subsection*{Program committee member}

CHI ('19, '20, '21), CSCW '18, GROUP '23, PervasiveHealth ('16, '15), Mindcare '16
% And in workshops at CHI '20, ICEC'18, CHI'16, GROUP '14, Ubicomp '13, SouthCHI '13

% And several workshops:\\
% \years{2016}Advances in DIY Health \& Wellbeing at CHI '16\\
% \years{2014}Collaboration and Coordination in the Context of Informal Care Workshop at GROUP '14\\
% \years{2013}2nd Workshop on Computer Mediated Social Offline Interactions at Ubicomp '13\\
% \years{2013}Design Culture for Ageing Well Workshop at SouthCHI '13

\subsection*{Workshop organizer}

\years{2022}Revisiting Patient-Clinician Interaction in 2022: Challenges from the Field and Opportunities for Future Research at ECSCW '22 (\href{http://francisconunes.me/RevisitingPatientClinicianInteractionWS/index.html}{link})\\
\years{2021}Realizing AI in Healthcare: Challenges Appearing in the Wild at CHI '21 (\href{http://francisconunes.me/RealizingAIinHealthcareWS/index.html}{link})\\
\years{2020}Dis-entangling later life: ageing processes, innovative practices \& critical reflections? at STS Italia '20\\
\years{2019}Who Cares? Exploring the Concept of Care Networks for Designing Healthcare Technologies at ECSCW '19 (\href{http://francisconunes.me/CareNetworksWS/}{link})\\
\years{2014}Designing Self-care for Everyday Life at NordiCHI '14 (\href{https://designingselfcareforeverydaylife.wordpress.com/}{link})
%\years{2014}Designing Systems for Health and Entertainment: what are we missing? at ACE '14 (\href{https://designingsystemsforhealthandentertainment.wordpress.com/}{link})

\subsection*{Reviewer}

Health Informatics Journal, IwC, IJCCI, IJHCS, IxD\&A, JAMIA, JCSCW, JMIR, TOCHI.\\
CHI ('14-'23), CSCW ('15-'22), DIS '18, GROUP ('16, '23), Healthcom '13, Interact '19, MobileHCI '17, NordiCHI ('14, '16), PervasiveHealth ('15-'17).

\subsection*{Student volunteer}

CHI ('13, '16), CSCW '15 PC meeting, NordiCHI '10

%\break




%----------------------------------------------------------------------------------------
%	PUBLICATIONS AND TALKS SECTION
%----------------------------------------------------------------------------------------

\section*{Publications \& talks}

h-index: 16; citations: 1093 (source Google Scholar as of 02/2024)

\subsection*{Proceedings and special issues edited}

\years{2023}T. O. Andersen, F. Nunes, L. Wilcox, E. Coiera, Y. Rogers (2023) ``Special Issue on Human-Centred AI in Healthcare: Challenges Appearing in the Wild'', \emph{ACM Transactions on Computer Human Interaction}, 30(2).\\
\years{2016}P. A. Silva, F. Nunes, eds. (2016) ``Special issue on fun and engaging computing technologies for health'', \emph{Entertainment Computing}, 15: 41--70.\\
\years{2015}N. Verdezoto, F. Nunes, E. Grönvall, G. Fitzpatrick, C. Storni, M. Kyng, eds. (2015) ``Proceedings of Designing Self-care for Everyday Life. Workshop in conjunction with NordiCHI 2014, 27th October 2014, Helsinki, Finland'', \emph{DAIMI Report Series}, 41(597).\\
\years{2014}N. Verdezoto, F. Nunes, G. Fitzpatrick, E. Grönvall, eds. (2014) ``Designing Self-Care For Everyday Life'', \emph{Interaction Design and Architecture(s) Journal - IxD\&A}, 23: 142--190.

\subsection*{Journal articles}
\years{2023}E. Kuosmanen, E. Huusko, N. van Berkel, F. Nunes, J. Vega, J. Goncalves, M. Khamis, A. Esteves, D. Ferreira, S. Hosio (2023) ``Exploring crowdsourced self-care techniques: A study on Parkinson's disease'', \emph{International Journal of Human-Computer Studies}, 177, 103062.\\
\years{2023}F. Nunes, P. Rato Grego, R. Araújo, P. A. Silva (2023) ``Self-report user interfaces for patients with Rheumatic and Musculoskeletal Diseases: App review and usability experiments with mobile user interface components'', \emph{Computers \& Graphics}, 117, 61--72.\\
\years{2023}S. Rego, A. R. Henriques, S. Silvério Serra, T. Costa, A. M. Rodrigues, F. Nunes (2023) ``Methods for the Clinical Validation of Digital Endpoints: Protocol for a Scoping Review'', \emph{JMIR Research Protocols}, 2023;12:e47119.\\
\years{2023}S. Rêgo, M. Dutra-Medeiros, F. Nunes (2023) ``The Challenges of Setting Up a Clinical Study with the New European Union Medical Device Regulation'', \emph{Acta Médica Portuguesa}, 2023 Jul-Aug;36(7-8):455--457.\\
\years{2022}C. Mendes-Santos, F. Nunes, E. Weiderpass, R. Santana, G. Andersson (2022) ``Development and Evaluation of the Usefulness, Usability, and Feasibility of iNNOV Breast Cancer: Mixed Methods Study'', \emph{JMIR Cancer} 8(1):e33550.\\
\years{2022}C. Mendes-Santos, F. Nunes, E. Weiderpass, R. Santana, G. Andersson (2022) ``Understanding Mental Health Professionals' Perspectives and Practices Regarding the Implementation of Digital Mental Health: Qualitative Study'', \emph{JMIR Formative Research} 6 (4), e32558.\\
\years{2022}F. Miele, F. Nunes (2022) ``Has COVID-19 Changed Everything? Exploring Turns in Technology Discourses and Practices Related to Ageing'', \emph{TECNOSCIENZA} 12(2), 61-68.\\
\years{2022}S. Rêgo, M. Monteiro-Soares, M. Dutra-Medeiros, F. Soares, C. C. Dias, F. Nunes (2022) ``Implementation and Evaluation of a Mobile Retinal Image Acquisition System for Screening Diabetic Retinopathy: Study Protocol'', \emph{Diabetology} 3(1), 1-16.\\
\years{2021}F. Nunes, P. Madureira, S. Rêgo, C. Braga, R. Moutinho, T. Oliveira, F. Soares (2021) ``A Mobile Tele-Ophthalmology System for Planned and Opportunistic Screening of Diabetic Retinopathy in Primary Care'', \emph{IEEE Access}, 9: 83740-83750.\\
\years{2020}J. Almeida, F. Nunes (2020) ``The Practical Work of Ensuring Effective Use of Serious Games in a Rehabilitation Clinic: A Qualitative Study'', \emph{JMIR Rehabilitation and Assisted Technologies - JRAT}, 7(1).\\
\years{2017}F. Nunes, T. Andersen, G. Fitzpatrick (2017) ``The agency of patients and carers in medical care and self-care technologies for interacting with doctors'', \emph{Health Informatics Journal}, 25(2): 330--349.\\
\years{2016}F. Güldenpfennig, F. Nunes, G. Fitzpatrick (2016) ``Making Space to Engage: An Open-ended Exploration of Technology Design with Older Adults'', \emph{International Journal of Mobile Human Computer Interaction}, 8(2): 1--19.\\
\years{2016}F. Nunes, P. A. Silva, J. Cevada, A. C. Barros, L. Teixeira (2015) ``User Interface Design Guidelines for Smartphone Applications for People With Parkinson's Disease'', \emph{Universal Access in the Information Society}, 15(4): 659--679.\\
\years{2015}F. Nunes, N. Verdezoto, G. Fitzpatrick, M. Kyng, E. Grönvall, C. Storni (2015) ``Self-care Technologies in HCI: Trends, Tensions, and Opportunities'', \emph{ACM Transactions on Computer Human Interaction}, 22(6), article 33: 1--45.\\
\years{2015}F. Nunes, G. Fitzpatrick (2015) ``Self-care Technologies and Collaboration'', \emph{International Journal of Human Computer Interaction}, 31(12): 869--881.\\
\years{2012}A. Aguiar, F. Nunes, M. Silva, P. A. Silva, D. Elias (2012) ``Leveraging electronic ticketing to provide personalised navigation in a public transport network'', \emph{IEEE Transactions on Intelligent Transportation Systems}, 13(1): 213--220.


\subsection*{Long conference papers (peer-reviewed)}
\years{2023}B. Ramalho, S. Gama, F. Nunes (2023) ``Patient Dashboards of Electronic Health Record Data to Support Clinical Care: A Systematic Review'', \emph{Proc. of ICHI}, 407--419.\\
\years{2021}F. Nunes, J. Almeida, C. Chung, N. Verdezoto (2021) ``Avoiding Reactions Outside the Home: Challenges, Strategies, and Opportunities to Enhance Dining Out Experiences of People with Food Hypersensitivities'', \emph{Proc. of CHI 2021}, 208:1--208:16.\\
\years{2019}A. Pereira, D. Folgado, F. Nunes, J. Almeida, I. Sousa (2019) ``Using Inertial Sensors to Evaluate Exercise Correctness in Electromyography-based Home Rehabilitation Systems'', \emph{Proc. of MEMEA 2019}.\\
\years{2019}J. Silva, D. Gomes, F. Nunes, D. Moreira, J. Alves, A. Pereira, I. Sousa (2019) ``Position-independent Physical Activity Monitoring: Development and Comparison with Market Devices'', \emph{Proc. of MEMEA 2019}.\\
\years{2018}F. Nunes, G. Fitzpatrick (2018) ``Understanding the mundane nature of self-care: Ethnographic accounts of people living with Parkinson's'', \emph{Proc. of CHI 2018}, 402:1--402:15.\\
\years{2018}F. Nunes, J. Ribeiro, C. Braga, P. Lopes``Supporting the Self-care Practices of Shift Workers'', \emph{Proc. of MUM 2018}, 71--81.\\
\years{2018}J. Silva, E. Oliveira, D. Moreira, F. Nunes, M. Caic, J. Madureira, Eduardo Pereira (2018) ``Design and evaluation of a fall prevention multiplayer game for senior care centres'', \emph{Proc. of ICEC 2018}, 103--114.\\
\years{2013}P. Jordan, P. A. Silva, F. Nunes, R. Oliveira (2013) ``mobileWAY - A System to Reduce the Feeling of Temporary Lonesomeness of Persons with Dementia and to Foster Inter-caregiver Collaboration'', \emph{Proc. of HICSS 2013}, 2474--2483.\\
\years{2013}P. A. Silva, F. Nunes, A. Vasconcelos, M. Kerwin, R. Moutinho, P. Teixeira (2013) ``Using the Smartphone Accelerometer to Monitor Fall Risk While Playing a Game: The Design and Usability Evaluation of Dance! Don't Fall'', \emph{Proc. of HCII 2013}, 754--763.\\
\years{2012}F. Nunes, M. Kerwin, P. A. Silva (2012) ``Design Recommendations for TV User Interfaces for Older Adults: Findings From the eCAALYX Project'', \emph{Proc. of ASSETS 2012}, 41--48.\\
\years{2012}M. Kerwin, F. Nunes, P. A. Silva (2012) ``Dance Don’t Fall -- Preventing Falls and Promoting Exercise at Home'', \emph{Proc. of pHealth 2012}, 254--259.\\
\years{2012}A. Vasconcelos, P. A. Silva, J. Caseiro, F. Nunes, L. F. Teixeira (2012) ``Designing Tablet-based Games for Seniors: the Example of Cogniplay, a Cognitive Gaming Platform'', \emph{Proc. of Fun \& Games 2012}, 1--10.\\
\years{2010}P. A. Silva, F. Nunes (2010) ``3 x 7 Usability Testing Guidelines for Older Adults'', \emph{Proc. MEXIHC 2010}, 2: 1--8.\\
\years{2010}F. Nunes, P. A. Silva, F. Abrantes (2010) ``Human-Computer Interaction and the Older Adult: an Example Using User Research and Personas'', \emph{Proc. of PETRA 2010}, 49: 1--8.


\subsection*{Short conference papers (peer-reviewed)}

\years{2023}C. Nave, F. Nunes, T. Romão, N. Correia (2023) ``Exploring Emotions: Study of Five Design Workshops for Generating Ideas for Emotional Self-report Interfaces'', \emph{Proc. of INTERACT 2023}.\\
\years{2020}C. Braga, S. Rego, F. Nunes (2020) ``Clinicians' Perspectives on Using Mobile Eye Fundus Cameras to Screen Diabetic Retinopathy in Primary Care'', \emph{Proc. of ICHI 2020}.\\
\years{2018}A. Vasconcelos, F. Nunes, A. Carvalho, C. Correia (2018) ``Mobile, Exercise-agnostic, Sensor-based Serious Games for Physical Rehabilitation at Home'', \emph{Proc. of TEI 2018}, 271--278.\\
\years{2018}V. Guimarães, F. Nunes (2018) ``A Wearable, Customizable, and Automated Auditory Cueing System to Stimulate Gait in Parkinson's'', \emph{Proc. of PervasiveHealth 2018}.\\
\years{2018}A. Pereira, F. Nunes (2018) ``Physical Activity Intensity Monitoring of Hospital Workers using a Wearable Sensor'', \emph{Proc. of PervasiveHealth 2018}.\\
\years{2018}R. Peixoto, J. Ribeiro, E. Pereira, F. Nunes, A. Pereira (2018) ``Designing the Smart Badge: A Wearable Device for Hospital Workers'', \emph{Proc. of PervasiveHealth 2018}.\\
\years{2018}F. Aluvathingal, N. Verdezoto, F. Nunes (2018) ``Exploring Challenges and Opportunities to Support People with Food Allergies to Find Safe Places for Eating Out'', \emph{Proc. of 32nd British HCI Conference}.\\
\years{2017}F. Güldenpfennig, Ö.Subasi, F. Nunes, M. Urbanek (2017) ``UbiKit: Learning to Prototype for Tangible and Ubiquitous Computing'', \emph{31st British HCI Conference}.\\
\years{2015}F. Güldenpfennig, F. Nunes, G. Fitzpatrick (2015) ``Proxycare: Integrating informal care into formal settings'', \emph{Proc. of PervasiveHealth 2015}, 141--144.\\
\years{2011}T. Marques, F. Nunes, P. A. Silva, R. Rodrigues (2011) ``Tangible Interaction on Tabletops for Elderly People'', \emph{Proc. ICEC 2011}, pp. 440--443.\\
\years{2011}T. Sousa, P. Tenreiro, P. A. Silva, F. Nunes, E. Rodrigues (2011) ``Cross-platform social web application for older adults with html 5'', \emph{Proc. ICEC 2011}, pp. 375--378.

\subsection*{Position papers (juried)}
\years{2023}F. Nunes, J. Couto Silva, B. Félix, R. Melo, H. Winschiers-Theophilus, N. Bagalkot, N. Verdezoto, S. Lazem, A. Van Heerden, T. Ngubane, S. Till, M. Densmore (2023) ``African Co-Design: Past, Present, and Emerging'', \emph{Proc. of AfriCHI'23}.\\
\years{2023}T. O. Andersen, F. Nunes, L. Wilcox, E. Coiera, Y. Rogers (2023) ``Introduction to the Special Issue on Human-Centred AI in Healthcare: Challenges Appearing in the Wild'', \emph{ACM Transactions on Computer Human Interaction}, 30 (2): article 25.\\
\years{2023}R. Araújo, P. Matias, P. Studenic, M. Valada, R. Graca, N. Nakhost Lotfi, D. Belo, F. Nunes (2023)``POS0978 Exploring smartphone-based digital endpoints for rheumatic conditions'', \emph{Annals of the Rheumatic Diseases}, 2023;82:805.\\
\years{2023}F. Nunes, C. Correia (2023) ``HomeSenseALS: A mobile app for people with Amyotrophic Lateral Sclerosis'', \emph{Proc. of WISH at CHI 2023}.\\
\years{2022}F. Nunes, N. Verdezoto, T. Andersen, S. Matthiesen, C. Chung, S. Y. Park, W. Seo, P. Studenic (2022) ``Revisiting Patient-Clinician Interaction in 2022: Challenges from the Field and Opportunities for Future Research'', \emph{Proc. of ECSCW 2022}.\\
\years{2021}F. Nunes, J. Silva (2021) ``Monitoring patients with rheumatic conditions remotely using data from smartphone'', \emph{Book of abstracts 3rd Digital Rheumatology Day}.\\
\years{2021}T. O. Andersen, F. Nunes, L. Wilcox, E. Kaziunas, S. Matthiesen, F. Magrabi (2021) ``Realizing AI in Healthcare: Challenges Appearing in the Wild'', \emph{Proc. of CHI EA 2021}.\\
\years{2020}C. Correia, E. Oliveira, F. Nunes (2020) ``Using Illustration to Create More Inclusive User Interfaces for Older Adults'', \emph{Interactions}, 27(2), 79--81.\\
\years{2019}F. Nunes (2019) ``From medicalized to mundane self-care technologies'', \emph{Interactions}, 26(3), 67--69.\\
\years{2019}S. Y. Park, F. Nunes, A. Berry, A. Büyüktür, L. de Russis, M. Czerwinski, W. Seo (2019) ``Who Cares? Exploring the Concept of Care Networks for Designing Healthcare Technologies'', \emph{Proc. of ECSCW 2019}.\\
\years{2016}P. A. Silva, F. Nunes (2016) ``Preface for the special issue on fun and engaging computing technologies for health'', \emph{Entertainment Computing}, 15: 41--42.\\
\years{2016}F. Nunes (2016) ``The medication reminder as enforcing a medicalised perspective on daily life'', \emph{Book of abstracts of the 6th STS Italia Conference}, 58.\\
\years{2015}F. Nunes (2015) ``Designing Self-care Technologies for Everyday Life: A Practice Approach'', \emph{Proc. of CHI EA 2015}, 215--218.\\
\years{2014}N. Verdezoto, F. Nunes, G. Fitzpatrick, E. Grönvall (2014) ``Preface to the Focus Section'', \emph{Interaction Design and Architecture(s) Journal - IxD\&A}, 23.\\
\years{2013}F. Nunes (2013) ``Improving the Self-care of Parkinson’s Through Ubiquitous Computing'', \emph{Proc. of Ubicomp Adjunct 2013}.\\
\years{2011}F. Nunes and, P. A. Silva (2011) ``The Smartphone -- A Mobile Companion for Older Adults'', \emph{Proc. of Promoting and Supporting Healthy Living by Design WS at INTERACT 2011}, 41--43.\\
\years{2011}P. M. Teixeira, P. A. Silva, F. Nunes, L. F. Teixeira (2011) ``Mover - Activity Monitor and Fall Detector for Android'', \emph{Proc. of Mobile Wellness Workshop at MobileHCI 2011}.\\
\years{2010}F. Nunes, P. A. Silva (2010) ``Fostering Wellness of Older Adults While Performing Usability Testing'', \emph{Proc. of HCI4WELL Workshop at HCI2010}.\\
\years{2009}A. Aguiar, F. Nunes, M. Silva, D. Elias (2009) ``Personal Navigator for a Public Transport System Using RFID Ticketing'', \emph{Proc. of inMotion09 Workshop at Pervasive 2009}.

\subsection*{Invited presentations}

\years{24/02/2022}``Accompanying participants during field trials, at a distance" at University of Copenhagen, Copenhagen, Denmark.\\
\years{23/02/2022}``Self-care technologies – an overview of technologies and opportunities" at University of Copenhagen, Copenhagen, Denmark.\\
\years{10/12/2021}``Accompanying participants during field trials, at a distance" at University of Lisbon, Lisbon, Portugal.\\
\years{23/10/2021}``Avoiding reactions outside the home'' at University of Porto, Porto.\\
\years{24/09/2021}``Accompanying participants during field trials, at a distance'' at University College of Dublin, Dublin, Ireland.\\
\years{04/06/2021}``Avoiding reactions outside the home'' at University of Coimbra, Coimbra, Portugal.\\
\years{01/06/2021}``Self-care technologies'' at Escola Superior de Saúde, Porto, Portugal.\\ % Course of Project Seminar in Health, for the EMMaH 2020-21 
\years{09/09/2019}``Towards Mundane Personal Assistants'' Keynote at Ubicomp Workshop on Ubiquitous Personal Assistance, London, UK.\\
\years{03/05/2019}``Technology Assessment in Real Life'' at Escola Superior de Saúde, Porto, Portugal.\\ % II International Meeting of Medical Technology and Healthcare Business
\years{14/09/2015}``Understanding and Promoting the Creation of Patient Knowledge Online'' at University of Trento \& FBK, Italy.\\
\years{17/08/2015}``How can technology help people improve their self-care? Supporting engagement in their clinical consultations'' at IT University of Copenhagen, Denmark.\\
\years{17/08/2015}``Designing Self-care Technology for People with Parkinson’s: A Focus on Everyday Life'' at University of Copenhagen, Denmark.\\
% \years{12/07/2012}``Design recommendations for TV user interfaces for older adults'' at the Interactive Assistive Technologies for the Elderly Workshop in University of Aveiro, Portugal.\\
% \years{06/05/2011}``eCAALYX project'' at the course Multimodal Interfaces of University of Porto, Portugal.

% \subsection*{Media mentions}

% Mover was mentioned in the \href{http://www.jornaldenegocios.pt/empresas/detalhe/evitar_acidentes_com_idosos_pode_estar_agrave_distacircncia_de_um_telemoacutevel.html}{Jornal de Negócios} (30/04/2010) and \href{http://www.dn.pt/inicio/ciencia/interior.aspx?content_id=1581122&seccao=Tecnologia}{Diário de Notícias} (29/05/2010).\\
% Naviporto was news in \href{http://www.jn.pt/paginainicial/interior.aspx?content_id=1576870}{Jornal de Notícias} (24/05/2010) and \href{http://www.dn.pt/inicio/ciencia/interior.aspx?content_id=1581122&seccao=Tecnologia}{Diário de Notícias} (29/05/2010).\\
% Smart Companion was described in \href{https://www.youtube.com/watch?v=nWqpXHaN5Is}{RTP} (01/08/2011), \href{https://www.youtube.com/watch?v=gIa2ZMeJ7MI}{Porto Canal}, and another time at \href{https://www.youtube.com/watch?v=miAXlLm2GSQ}{Porto Canal} (26/07/2012) when I was even interviewed.	

\section*{Languages}

Portuguese -- native\\
English -- proficient\\
Spanish -- fluent\\
French -- intermediate (DELF A4 certificate)\\
German -- intermediate (B1.1 certificate)\\

\section*{References}

Geraldine Fitzpatrick, TU Wien (\href{mailto:geraldine.fitzpatrick@tuwien.ac.at}{geraldine.fitzpatrick@tuwien.ac.at})\\
Ana Correia de Barros, Fraunhofer Portugal AICOS
(\href{mailto:ana.barros@fraunhofer.pt}{ana.barros@fraunhofer.pt})\\
Paula Alexandra Silva, University of Coimbra (\href{mailto:palexa@gmail.com}{palexa@gmail.com})\\
Nuno Correia, Universidade NOVA de Lisboa (\href{mailto:nmc@fct.unl.pt}{nmc@fct.unl.pt})\\


% Projects
% Mover http://www.youtube.com/watch?v=mut3Ghw3hF8

% \vfill{} % Whitespace before final footer

%----------------------------------------------------------------------------------------
%	FINAL FOOTER
%----------------------------------------------------------------------------------------

\begin{center}
{\scriptsize Last updated: \today\ } % Any final footer text such as a URL to the latest version of your CV, last updated date, compiled in XeTeX, etc
\end{center}

%----------------------------------------------------------------------------------------

\end{document}